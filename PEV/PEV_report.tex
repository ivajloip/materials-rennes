\documentclass[12pt]{article}
\usepackage[utf8]{inputenc}
\usepackage[english,bulgarian]{babel}
\usepackage{amsfonts}
\usepackage{amsthm}
\usepackage{amsmath}
\usepackage{amscd}
\usepackage{amssymb}
\usepackage{graphics}
\usepackage{graphicx}
\usepackage{fancyhdr}
\usepackage{multicol}
\usepackage{listings}
\usepackage{xcolor}

\oddsidemargin 0mm
\evensidemargin 0mm
\topmargin 0mm
\textheight 216mm
\textwidth 165mm

\pagestyle{fancy}
\fancyhf{}
\fancyhead[CE,CO]{PEV report}
\fancyfoot[CE,CO]{\thepage}

\lstset{basicstyle=\ttfamily,
  mathescape=true,
  escapeinside=||,
  emph={Task, Operation, for, each, if, else, endif, endfor, while, then, When, do, repeat, until, endrepeat},
  emphstyle={\color{gray}\bfseries\itshape}}

\def\CC {{\mathbb C}}        % комплексни числа
\def\RR {{\mathbb R}}        % реални числа
\def\ZZ {{\mathbb Z}}        % цели числа
\def\NN {{\mathbb N}}        % естествени числа
\def\be  {\begin{eqnarray}}  % формула с номерация
\def\ee  {\end{eqnarray}}    % край на формулата
\def\ben {\begin{eqnarray*}} % формула без номерация
\def\een {\end{eqnarray*}}   % край на \bena
\newcommand{\hr}{\rule{\linewidth}{0.1mm}}
\newcommand{\bighs}{\hspace{15pt}}
\newcommand{\hs}{\hspace{10pt}}
\renewcommand{\tilde}{\overset{-}}

\newenvironment{itemize*}{
  \begin{itemize}
    \setlength{\itemsep}{0pt}
    \setlength{\parskip}{0pt}
    \setlength{\parsep}{0pt}
}{
  \end{itemize}
}

\newenvironment{enumerate*}{
  \begin{enumerate}
    \setlength{\itemsep}{0pt}
    \setlength{\parskip}{0pt}
    \setlength{\parsep}{0pt}
}{
  \end{enumerate}
}

%\renewenvironment{proof}[1][\proofname]{}{\qed}

\newtheoremstyle{plain}{1pt}{0pt}{}{}{}{}{.5em}{\thmname{\textbf{#1}}:\thmnote{#3}}

\theoremstyle{plain}

% Прави формулите да са с нормален шрифт на текста (примерно дробите не стават по-малки)
\everymath{\displaystyle}

\begin{document}

\section*{\centering
  Performance evaluation report - 2013-2014 \\
  by\\
  Ivaylo Petrov and Hristina Hristova 
}

\hr
\section{\textbf{System M \textbackslash M \textbackslash *}}

We are going to analyse a system with three \emph{M\textbackslash M\textbackslash 1}
servers and one \emph{M\textbackslash M\textbackslash 1\textbackslash H} server
as shown on \emph{Figure 1} by using the data from the simulator which we computed
for this system. From server 1 a unit can go to server 2 with a probability
\emph{p} and to server 3 - with probability \emph{1-p}. Since server 3 has a
limit of \emph{H} for the units in its waiting queue, there can occur losses
in the system.\\
some figure \\
For that system we have six free parameters: $lambda, mu_1, mu_2, mu_3, mu_4,
queue\_limit$. By varying them we will show the impact they have on the system
losses, the waiting times and the response times of the units.

\subsection*{Analysis}

Firstly, we have conducted experiments in which we varied separately the values of
the free parameters in the range of 1 to 100. This way we hope to find connections
between the behavior of these parameters and the rest of the variables. We have
also computed the correlations between the variables. It can be seen from
(\emph{Figure 2}) that there is little point in trying to find connections
between $mu_3$ and the response time of server 1 for example since they are
weakly correlated and because of pure logical reasons.\\
The simulation data is simulated with a parameter that varies while the rest of
the free parameters are constant (set to 40). In order to compare the results by
different criteria we have simulated each case twice - with 10 and then with 40
units limit for the third server when all but the queue limit is varied, and with
10 and then 40 as a value for the constant parameters when $queue\_limit$ is
being increased. The probability \emph{p} is set to 0.5.\\
We are interested in the waiting times of a unit in servers 3 and 4. The results
from varying $mu_1$ are shown in the form of graphics. For the waiting time in
server 3 we observe a local (most probably a global) maximum. It is interesting
that for small values of $mu_1$ ($\le$ 15) that waiting time is the same as when
$mu_1 \ge$ 90. \emph{Graphic 1} show also that when the limit of the queue of
server 3 increases, even smaller waiting times for bigger values of $mu_1$ are
observed. From the second graphic we can see how the losses of the system are
explained by $mu_1$. We conclude that the bigger $mu_1$ the bigger the lost
units.\\

\includegraphics[width=160mm, height=90mm]{graphics/chart_1.png}\\
\includegraphics[width=160mm, height=90mm]{graphics/chart_2.png}\\

So far we have observed that a system with parameters ($mu_1$, 40, 40, 40,
40/10) behaves well (in a sense that the losses are small) when $mu_1$ is small.
On the other hand the smaller the $mu_1$ the smaller the response time of a unit
for server 1.\\
Let us now explain the results we get from varying $mu_3$. For the waiting time
of server 3 we can notice that when $mu_3$ becomes $\ge queue\_limit$ it begins
to incline to 0. Still we observe that for smaller values of the limit of the queue
of server 3 the time a unit has to wait is smaller. Of course this depends on the
value of $mu_3$ as well which is set to 40. The response time of server 3 will be
low if the rate $mu_3$ is at high values. An interesting observation is the behavior
of the waiting time of server 4 when the queue limit is changing. Graphic 3 leads
as to a conclusion that firstly, the rate of the waiting time of server 4 is almost
constant (with a small dispersion) and secondly it is smaller for smaller values
of \emph{queue\_limit}.\\

\includegraphics[width=160mm, height=90mm]{graphics/chart_3.png}\\

Now we look at the loss rate and notice that at high levels of $mu_3$ losses are
not observed. Here we also observe a better behavior when the value of the queue
limit is 10 from when it is 40.\\
Now we will explain more with more details why and how we can manage to keep the
levels of the lost units in the system at low values of \emph{queue\_limit}. As
we can see from \emph{Graphic 5} when the limit is approximately 4 times less
than the service rate $mu_1$ the losses begin to incline to 0. This is very
interesting result because it means that we reduce the loss rate even when the
limit of the queue for server 3 is small. We do not need to set it to very high
levels in order to avoid losses.\\

\includegraphics[width=160mm, height=90mm]{graphics/chart_4.png}\\

\section*{\textbf{Cubic regression model}
}

By running our simulation and by varying the parameters, we firstly collected
diverse data which included over 2500 examples. The main idea is to build a model
based on this data that predicts the value of the losses in server 3 by taking
into account the parameters we vary. After making some calculations of the impact of the
free parameters on the variable \emph{losses server 3}, we concluded that the
best approach would be to use cubic regression. Our decision is supported by the
results which show that $m_3$ has the greatest influence on \emph{losses
server 3} and the cubic estimation of $mu_3$ is the best approximation for
the losses which occur in server 3.

\subsection*{The model}

In order to apply the regression approach to the data we firstly have to make some
assumptions. The free variables have to be independent (a.k.a uncorrelated). On
theory we can explain why they have to be independent very simply: since each
server has its own serving speed, they do not depend either on one another or on
the input rate \emph{lambda}. The results on the data from the simulations show
the same thing - $mu_1, mu_2, mu_3, mu_4, lambda, queue limit$ are uncorrelated.
This also proves that the way we have built the simulator is correct and the
influence of the randomness is reduced to minimal.\\
We have to assume that the residuals of the regression are normally distributed
and to check for this after building the model so that we can decide if it good
or not.\\
Before showing the results from the regression, we will introduce the reader to
the descriptive statistics of the free variables.\\
\emph{some data and explanations}\\
With the data given, we introduce the cubic regression model:\\

\begin{lstlisting}[frame=single]
  losses in server 3 := 399,446 -
                        0.052 * lambda +
                        1,515 * $mu1$ -
                        0.074 * $mu_2$ -
                        30,351 * $mu_3$ -
                        0,042 * $mu_4$ -
                        1,112 * queue limit +
                        0,304 * $mu_3^{2}$ -
                        0,001 * $mu_3^{3}$
\end{lstlisting}

The results show that when we know the values of the free parameters we can find
the size of the losses in server 3. The model explains 73\% of the data. This
results are very good given the number of observations. To see of the model is
correct we look at the residuals. As the graph shows, they are approximately
normally distributed. We conclude that in the 73\% of the cases the cubic regression
is the approach that explains the losses occurred in the system.

\end{document}
