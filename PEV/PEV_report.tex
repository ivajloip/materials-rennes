\documentclass[12pt]{article}
\usepackage[utf8]{inputenc}
\usepackage[english]{babel}
\usepackage{amsfonts}
\usepackage{amsthm}
\usepackage{amsmath}
\usepackage{amscd}
\usepackage{amssymb}
\usepackage{graphics}
\usepackage{graphicx}
\usepackage{fancyhdr}
\usepackage{multicol}
\usepackage{listings}
\usepackage{xcolor}
%\usepackage{enumitem}

\oddsidemargin 0mm
\evensidemargin 0mm
\topmargin 0mm
\textheight 216mm
\textwidth 165mm

\pagestyle{fancy}
\fancyhf{}
\fancyhead[CE,CO]{PEV report}
\fancyfoot[CE,CO]{\thepage}

\lstset{basicstyle=\ttfamily,
  mathescape=true,
  escapeinside=||,
  emph={Task, Operation, for, each, if, else, endif, endfor, while, then, When, do, repeat, until, endrepeat},
  emphstyle={\color{gray}\bfseries\itshape}}

%\setlist[itemize]{noitemsep,nolistsep} % remove spaces between itemize and
  % its surroundings and between its items
%\setlist[enumerate]{noitemsep,nolistsep} % remove spaces between enumerate and
  % its surroundings and between its items

\def\CC {{\mathbb C}}        % Complex numbers
\def\RR {{\mathbb R}}        % Real numbers
\def\ZZ {{\mathbb Z}}        % Integers
\def\NN {{\mathbb N}}        % Positive integers
\def\be  {\begin{eqnarray}}  % Begin of math formula with number
\def\ee  {\end{eqnarray}}    % End of math formula with number
\def\ben {\begin{eqnarray*}} % Begin of math formula without number 
\def\een {\end{eqnarray*}}   % End of math formula with number 

\newcommand{\hr}{\rule{\linewidth}{0.1mm}}
\newcommand{\bighs}{\hspace{15pt}}
\newcommand{\hs}{\hspace{10pt}}
\renewcommand{\tilde}{\overset{-}}

\newenvironment{itemize*}{
  \begin{itemize}
    \setlength{\itemsep}{0pt}
    \setlength{\parskip}{0pt}
    \setlength{\parsep}{0pt}
}{
  \end{itemize}
}

\newenvironment{enumerate*}{
  \begin{enumerate}
    \setlength{\itemsep}{0pt}
    \setlength{\parskip}{0pt}
    \setlength{\parsep}{0pt}
}{
  \end{enumerate}
}

\newtheoremstyle{plain}{1pt}{0pt}{}{}{}{}{.5em}{\thmname{\textbf{#1}}:\thmnote{#3}}

\theoremstyle{plain}
\renewcommand\thesection{\Roman{section}.}
\renewcommand\thesubsection{\thesection\Roman{subsection}.}
\renewcommand\thesubsubsection{\thesubsection\Roman{subsubsection}.}

% Makes the size of the symbols in the formulas fixed (not getting smaller)
\everymath{\displaystyle}

\begin{document}

\section*{\centering
  Performance evaluation report - 2013-2014 \\
  by\\
  Ivaylo Petrov and Hristina Hristova 
}

\hr

In this report we will describe the results of the simulator we have built for
two different network topologies. After analysing the data, we will introduce a
model that explains it in the majority of cases. After that we will make compare
our results from the simulation and the model to the approximation of the expected
behavior of the network topologies.

\section*{\textbf{System M \textbackslash M \textbackslash *}}

\begin{figure}[h]
  \caption{Graph topology}
  \includegraphics[width=\textwidth, height=90mm]{graphics/simple_graph.png}\\
  \label{fig:simple_graph}
\end{figure}

We are going to analyse a system with three \emph{M\textbackslash M\textbackslash 1}
servers and one \emph{M\textbackslash M\textbackslash 1\textbackslash H} server
as shown on \emph{Figure~\ref{fig:simple_graph}} by using the data from the simulator which we computed
for this system. From server 1 a unit can go to server 2 with a probability
\emph{p} and to server 3 - with probability \emph{1-p}. Since server 3 has a
limit of \emph{H} for the units in its waiting queue, losses can occur in the
system. From servers 2 and 3, the units go to server 4 with probability 1.\\
For that system we have six free parameters: $\lambda, \mu_1, \mu_2, \mu_3, \mu_4,
queue\_limit$. By varying them we will show the impact they have on the system
losses, the waiting times and the response times of the servers.

\subsection*{Analysis}

Firstly, we have conducted a number of experiments in which we varied separately the values of
the free parameters in the range of 1 to 100. The number of the input units in
each experiment was 1000. We hope to find connections between the behavior of
these parameters and the rest of the variables, namely units losses, waiting times
and response times. We have also computed the correlations between the variables.
For example, we know that there is little point in trying to find connections
between $\mu_3$ and the response time of server 1 since they are weakly correlated.
This fact is very natural, however.\\
The simulation data is obtained with one parameter that varies and the rest of
the free parameters set to a constant value (10 or 40). In order to compare the results by
different criteria we have simulated each case twice - first with 10 and then with 40
units limit for the third server when all but the queue limit is varied, and with
10 and then 40 as a value for the constant parameters when $queue\_limit$ is
being varied. The probability \emph{p} in those simulations was set to 0.5.\\
In the following paragraph we are interested in the waiting times of a unit in servers 3 and 4. The results
from varying $\mu_1$ are shown in Figure~\ref{fig:influence_of_mu1_on_wt3_when_queue_limit_changes}. For the waiting time in
server 3 we observe a local (most probably also a global) maximum. It is an interesting
fact that for small values of $\mu_1$ ($\le$ 15) the waiting time is the same as when
$\mu_1 \ge$ 90. Figure~\ref{fig:influence_of_mu1_on_wt3_when_queue_limit_changes} also shows that when the limit of the queue of
server 3 increases, smaller waiting times for bigger values of $\mu_1$ are
observed. From Figure~\ref{fig:influence_of_mu1_on_system:losses} we can see how the losses of the system are
explained by $\mu_1$. We conclude that the bigger $\mu_1$ the bigger the lost
units.

\begin{figure}
  \caption{Influence of $\mu_1$ on WT3 when queue limit changes}
  \includegraphics[width=\textwidth, height=90mm]{graphics/chart_1.png}\\
  \label{fig:influence_of_mu1_on_wt3_when_queue_limit_changes}
\end{figure}
\begin{figure}
  \caption{Influence of $\mu_1$ on system losses}
  \includegraphics[width=\textwidth, height=90mm]{graphics/chart_2.png}\\
  \label{fig:influence_of_mu1_on_system:losses}
\end{figure}

So far we have observed that a system with parameters ($\mu_1$, 40, 40, 40,
40, 10) behaves well (in a sense that the losses are small) when $\mu_1$ is small.
On the other hand the smaller the $\mu_1$ the smaller the response time of a unit
for server 1.

Let us now explain another series of simulations where we vary $\mu_3$. Firstly, for the waiting time
of server 3 we can notice that when $\mu_3$ becomes $\ge queue\_limit$, it begins
to incline to 0. We also observe that for smaller values of the limit of the queue
of server 3, the time a unit has to wait is smaller. This is natural, as each
unit has to wait all the other units that are before it in the queue and with
smaller queue, fewer such units will exist. Of course waiting time depends on the
value of $\mu_1$ as well which is set to 40. Let us also note that the response time of server 3 will be
low if the rate $\mu_3$ has a high value. Another interesting observation is the behavior
of the waiting time of server 4 when the queue limit is changing.
Figure~\ref{fig:influence_of_mu3_on_wt3_when_queue_limit_changes} leads
us to the conclusion that firstly, the rate of the waiting time of server 4 is almost
constant (with a small dispersion) and secondly, it is smaller for smaller values
of \emph{queue\_limit}. The last point can be explained by the fact that smaller values
for \emph{queue size} lead to bigger number of losses, which in turn leads to
less units reaching server 4 and this way decreasing its waiting time.

\begin{figure}
  \caption{Influence of $\mu_3$ on WT3 when queue limit changes}
  \includegraphics[width=\textwidth, height=90mm]{graphics/chart_3.png}\\
  \label{fig:influence_of_mu3_on_wt3_when_queue_limit_changes}
\end{figure}

If we look at the loss rate, we can remark that at high levels of $\mu_3$ losses
are not observed. Here we also notice a good behavior of the system in terms of
wait time when the value of the queue limit is 40, but even better when it is
10.  

Now we will explain with more details why and how we can manage to keep the
levels of the lost units in the system low at low values of \emph{queue\_limit}.
As we can see from Figure~\ref{fig:influence_of_mu3_on_system:losses} when the
queue limit is 10 and $\mu_3$ is approximately 5 the losses begin to incline to 0
while when the queue limit is 40 this happens when $\mu_3$ reaches 20. And when
$\mu_3$ is small the losses are less in the case when the queue limit is
smaller. This is very interesting result because it means that we reduce
the loss rate even when the limit of the size queue for server 3 is small. We
do not need to set it to very high levels in order to avoid losses.  

\begin{figure}
  \caption{Influence of $\mu_3$ on system losses}
  \includegraphics[width=160mm, height=90mm]{graphics/chart_4.png}\\
  \label{fig:influence_of_mu3_on_system:losses}
\end{figure}

\subsection*{\textbf{Cubic regression model}
}

After the primary analysis we turn our attention to building a model from the
simulation data and later on comparing the results from that model to the
results from the simulation and those of the approximation that will be
introduced in the next section. By running our simulation and by varying the
parameters, we firstly collected diverse data which included over 2500
examples. The main idea was to build a model based on this data that predicts
the value of the losses in server 3 by taking into account the parameters we
vary. If we have a good model for the losses, we will be able to calculate the
loss probability, the mean throughput, the waiting time and the response time.

After making some calculations of the impact of the free parameters on the
variable \emph{losses server 3}, we concluded that the best approach would be
to use cubic regression. Our decision is supported by the results which show
that $\mu_3$ has the greatest influence on \emph{losses server 3} and the cubic
estimation of $\mu_3$ is the best approximation for these losses.

In order to apply the regression approach to the data we firstly have to make
some assumptions. The free variables have to be independent (i.e.
uncorrelated). On theory we can explain why they have to be independent very
simply: since each server has its own serving speed, these speeds do not depend
neither on one another nor on the input rate $\lambda$. The results of the
data from the simulations show the same thing -
$\mu_1, \mu_2, \mu_3, \mu_4, \lambda, \text{\emph{queue limit}}$ are uncorrelated.
This also proves that the way we have built the simulator is correct and the
influence of the randomness is reduced to minimal. 

We have to assume that the residuals of the regression are normally distributed
and to check for this after building the model so that we can decide if it is
good or not.

The descriptive statistics of the free variables can be found in the document
statistical\_results/descriptive\_statistics\_model\_summary.pdf. In the some document
the reader can find also the model summary (which includes the square R and
adjusted R) of the model we have build.


With the data given, we compute the cubic regression model. The software that is
use for the statistical model is \emph{SPSS}:\\

\clearpage

\begin{lstlisting}[frame=single]
  losses in server 3 := 399,446 -
                        0.052 * $\lambda$ +
                        1,515 * $\mu_1$ -
                        0.074 * $\mu_2$ -
                        30,351 * $\mu_3$ -
                        0,042 * $\mu_4$ -
                        1,112 * queue limit +
                        0,304 * $\mu_3^{2}$ -
                        0,001 * $\mu_3^{3}$
\end{lstlisting}

The results show that when we know the values of the free parameters we can find
the size of the losses in server 3. The model explains 73\% ($R^{2}$ = 0.730) of the data. This
results are very good given the number of observations. We can also consider them representative. 
To see if the model is correct we look at the residuals in the document statistical\_results/residuals.pdf.
As the graphic shows, they are approximately normally distributed. We conclude that in the 73\% of the cases the cubic
regression is the approach that explains the losses occurred in the system.

\subsection*{The approximations}

Now that we have the cubic regression model for the losses, we have only one
step left before comparing the results from the simulation to those of the
theory predictions - introducing the approximations. In this section we will
compute the formulas which we need for the comparison - the loss probability
$p_{loss}$, the mean throughput \emph{F}, the mean response time \emph{E(R)}
and the mean waiting time \emph{E(W)}.

\begin{lstlisting}[frame=single]
  for a server M\M\1\H
  with input rate $\lambda$ and service rate $\mu$:

  q := $\frac{\lambda}{\mu}$
  $p_{loss}$ := $q^{H}$((1-q)/(1-$q^{H+1}$))
  F := $\lambda(1 - p_{loss})$
  
  From Little we have: F E(R) = E(N)
\end{lstlisting}

We have a formula for E(N) and by using \emph{Little} we can easily calculate
E(R). Furthermore, we can estimate the losses of a server if we know the
probability $p_{loss}$ and the probability \emph{p} for a unit to go to that
server. In general, \emph{losses} = \emph{output * p} * $p_{loss}$, where
\emph{output} is the number of units that have been served in the previous
server, connected to the one we analyze. There is a way to compute the mean
waiting time too. We use the following reasoning: when we have the mean clients
in the server queue, we simply multiply it by the service time. This gives us a
good approximation for the mean waiting time.

Let us remark how we are going to compute the results for the cubic regression.
The model gives us the losses in the server and by using them, we can compute
the loss probability (as shown above). For the mean response time, the mean
throughput and the mean waiting time we use the formulas we have decided to use
for the approximations but with a loss probability that is equal to the one
estimated with the cubic regression.

\subsection*{Comparisons and results}

Let us first take a look at the behavior of the third server. In the file
\linebreak model\_comparisons/results\_sgraph\_cubic.pdf are the results which we got from the
simulation, from the cubic regression and from applying the formulas from the
theoretical approximation. We have done several examples varying the free
parameters and we will explain the observed results.  

The lighter green lines correspond to extremely good results shown by our
simulation. As it can be seen from them, the losses, the loss probability, the
mean throughput and the mean response errors are close to zero. In the first
examples we set the probability \emph{p} to 0.5. To understand whether \emph{p}
is the factor that makes our model correct, we change it to 0.7 and we observe
that the accuracy of the simulation is the same.

There are two cases, though, where we can see that the simulation gives worse
results than the results from the regression model. Firstly, let us look at the
example (200, 100, 10, 10, 10, 1) with the probability \emph{p} = 0.5. We can notice
that the results from the regression model are very close the approximations but not as
close as if the probability \emph{p} is slightly increased to 0.504. The fact
that the regression model we have build does not depend on the value of
\emph{p} is interesting (and on one hand proves its correctness) even if at
first look this may seem strange. If we see more closely the values of the
losses when \emph{p} is changing we will notice that they are the
same. Still, when \emph{p} is 0.504 the cubic approximation is better than the
results from the simulation.  This is because the real model depends strongly
on that probability and it is changing. For the majority of cases, it can be
concluded that when for some cases the cubic regression model is not strong
enough, there exists situation (when \emph{p} is different) in which the results
with the same loss levels are much better and closer to the observed by the approximation results
(even closer than the one observed by the simulation).

Surprisingly, in the case described above, the simulation does not just behave
worse than the cubic approximation, it behaves much worse than for example in
the other cases. This gives us a space for consideration whether this is true in
general. In the situations in which the simulation is a bad approximation we
may have built a model that can explain extremely well the expected results (the cubic regression).

Of course, cubic regression model has some disadvantages. Since it is not
absolutely accurate (it seems to overestimate the loss probability) there exist
situations in which the mean response time is negative. In these cases, we
cannot conclude that the cubic model is correct. Such cases are marked in
yellow in the document. Even more inadequate is the behavior of the regression
model in the case marked in dark red. The losses are negative which
cannot be interpreted logically.  Such distortions occur because $\mu_3$ is
significantly bigger than in the previous examples (and than the other $\mu$s
and $\lambda$) and this has a huge impact on the results of the cubic model. In
this case though, at least the simulation results are quite satisfying.

One question that needs to be answered is when the simulation behaves bad.
We have already seen one example of such a behavior. We can observe that
unlikely the other cases, that particular case has $\lambda$ that is bigger than
$\mu_1$. To see if this is the right conclusion, at the end we have shown more
examples with this property. The results from them are clear. We can conclude
that when $\lambda < \mu_1$ the simulation results are distorted.

Let us now look at another more complex topology as shown on the Figure~\ref{fig:hard_graph}.
We have one more server - server 5. This time the units from server 2 can go to
server 3 as well with a probability of \emph{1- $p_2$}. We have simulated the
behavior of such a graph and now we will introduce the results in comparison to
the expected values for the characteristics.

\begin{figure}
  \caption{Graph Topology}
  \includegraphics[width=\textwidth, height=90mm]{graphics/hard_graph.png}\\
  \label{fig:hard_graph}
\end{figure}

As a first example we take the following values for the free characteristics:
(10, $\mu_1$, 10, 10, 10, 10, 10) with probabilities \emph{1 - $p_1$} = 0.5 and
\emph{1 - $p_2$} = 0.3. We compute the approximation of the expected waiting
time of the third server and on one graphic we compare it to the results by the
simulation. Figure~\ref{fig:waiting_time1} shows that the levels of the waiting time that is to
be expected are slightly lower than these of our simulation. It is observed that
both times are increasing at the same speed. The difference of the waiting times
is not very big still more accuracy could have been achieved.\\
Now let us change the values of the parameters and make the same experiment. Now
the characteristics are (40, $\mu_1$, 40, 40, 40, 40, 40) with the same probabilities.
On the Figure~\ref{fig:waiting_time2} are presented the results. When $\mu_1$ is around and more than 20, it is observed the opposite in
this case - the values we get for the waiting time of server 3 from our simulation are
smaller than the expected ones. Here we notice bigger difference in the times. Again they
are increasing in the both situations but now the speed of the results from the approximation is
twice or more the speed of the results from the simulation for $\mu_1 \ge 20$.
It can be concluded that the simulation in the cases presented so far concerning
the more complex topology is behaving well when there is a constriction for $\mu_1$.\\

\begin{figure}
  \caption{Behavior of the simulation waiting time when $\mu_1$ increases}
  \includegraphics[width=160mm, height=90mm]{graphics/chart_5.png}\\
  \label{fig:waiting_time1}
\end{figure}

\begin{figure}
  \caption{Behavior of the simulation waiting time when $\mu_1$ increases - different parameters}
  \includegraphics[width=160mm, height=90mm]{graphics/chart_6.png}\\
  \label{fig:waiting_time2}
\end{figure}

We are not only considering waiting time but we are also interested in the mean
throughput especially in the cases when $\lambda$ and $\mu_2$ are changing.
The next two following figures show represent the mean throughput in these situations.
First we have the values (lambda, 40, 40, 40, 40, 40, 40) (the probabilities in
the both cases stay the same). The main observations are that the expected throughput
and the simulated one are increasing and with the increasing of $\lambda$ the
difference between them becomes bigger. This is explained with the different loss
probabilities - the expectation is that there will be losses but our simulation
does not detect any.\\
The more interesting case is when we vary $\mu_2$. This time our simulations
computes non-zero losses like the expectation follows. As shown on Figure~\ref{fig:throughput2}
for small values of $\mu_2$ the results from the simulation are quite close to
what is estimated as an approximation. With the increasing of $\mu_2$, the both
throughput seem to be almost constant. The difference between them is between 1
and 2 units. This is better behavior than in the previous case due to the fact
that the increasing speeds in the both cases are almost 0 for big $\mu_2$ and the
difference is constant.\\

\begin{figure}
  \caption{Behavior of the simulation mean throughput when $\lambda$ increases}
  \includegraphics[width=160mm, height=90mm]{graphics/chart_7.png}\\
  \label{fig:throughput1}
\end{figure}

\begin{figure}
  \caption{Behavior of the simulation mean throughput when $\mu_2$ increases}
  \includegraphics[width=160mm, height=90mm]{graphics/chart_8.png}\\
  \label{fig:throughput2}
\end{figure}


\end{document}
