\documentclass[12pt]{article}
\usepackage[utf8]{inputenc}
\usepackage[english,bulgarian]{babel}
\usepackage{amsfonts}
\usepackage{amsthm}
\usepackage{amsmath}
\usepackage{amscd}
\usepackage{amssymb}
\usepackage{graphics}
\usepackage{graphicx}
\usepackage{fancyhdr}
\usepackage{multicol}
\usepackage{listings}
\usepackage{xcolor}
\usepackage{enumitem}

\oddsidemargin 0mm
\evensidemargin 0mm
\topmargin 0mm
\textheight 216mm
\textwidth 165mm

\pagestyle{fancy}
\fancyhf{}
\fancyhead[CE,CO]{PEV Simulation}
\fancyfoot[CE,CO]{\thepage}

\lstset{basicstyle=\ttfamily,
  mathescape=true,
  escapeinside=||,
  emph={Task, Operation, for, each, if, else, endif, endfor, while, then, When, do, repeat, until, endrepeat},
  emphstyle={\color{gray}\bfseries\itshape}}

\setlist[itemize]{noitemsep,nolistsep}
\setlist[enumerate]{noitemsep,nolistsep}

\def\CC {{\mathbb C}}        
\def\RR {{\mathbb R}}       
\def\ZZ {{\mathbb Z}}      
\def\NN {{\mathbb N}}     
\def\be  {\begin{eqnarray}} 
\def\ee  {\end{eqnarray}}  
\def\ben {\begin{eqnarray*}} 
\def\een {\end{eqnarray*}}   
\newcommand{\hr}{\rule{\linewidth}{0.1mm}}
\newcommand{\bighs}{\hspace{15pt}}
\newcommand{\hs}{\hspace{10pt}}
\renewcommand{\tilde}{\overset{-}}

\newenvironment{itemize*}{
  \begin{itemize}
    \setlength{\itemsep}{0pt}
    \setlength{\parskip}{0pt}
    \setlength{\parsep}{0pt}
}{
  \end{itemize}
}

\newenvironment{enumerate*}{
  \begin{enumerate}
    \setlength{\itemsep}{0pt}
    \setlength{\parskip}{0pt}
    \setlength{\parsep}{0pt}
}{
  \end{enumerate}
}

\newtheoremstyle{plain}{1pt}{0pt}{}{}{}{}{.5em}{\thmname{\textbf{#1}}:\thmnote{#3}}

\theoremstyle{plain}

\everymath{\displaystyle}

\renewcommand{\labelenumii}{\arabic{enumi}.\arabic{enumii}}

\begin{document}

\section*{\centering
  Code explonation
}

\hr

The proposed simulation engine is entirely written in Python. It is tested with
python version 3.3, however it should work with older version as well.

The simulation engine can receive a number of parameters: 
\begin{itemize*}
  \item '-nc' $\rightarrow$ how many clients should be provided as input.
  \item '-queue-sizes' $\rightarrow$ a comma-separated list of the sizes for
    each of the queues. For the queues that are infinitely big, the user has to
    enter a value equal or greater than the number of clients.
  \item '-muis' $\rightarrow$ a comma-separated list of the speed for
    processing the clients for each server.
  \item '-lambdas' $ \rightarrow $ a comma-separated list of the speed at which
    the clients arrive to the input servers. Please note that the code was
    never tested with more than one input - server 1.
  \item '-graph' $ \rightarrow $ The file that describes the graph topology.
    That file contains the number of servers, input servers, output servers,
    and connections between servers with the probability to follow some of 
    those connection. For example is we have 0 2 0.5 this means that from 
    server 1 we can go to server 3 and the probability to go there, and not to
    some of the other servers that 0 is connected to, is 0.5.
  \item '-output-file' $ \rightarrow $ A file to which to append a
    comma-separated representation of the output (useful for processing with
    excel). It is optional.  
  \item '-repetitions' $ \rightarrow $ The number of execution of the simulation
    from which the mean statistics are extracted. If it is not provided, 100 is
    used.
\end{itemize*}

Those parameters are saved in an instance of the class \textbf{Configuration}
that is passed to the methods that need it.

For the purpose of this section when we say \textbf{event} we will refer to the
event of client packet being received by some of the servers or departing from
the servers.

The main idea of the simulation engine is to have a set of events and process 
them according to the time at which they occur. Each event has time when it
arises, type - arrival at server or departure, server to or from which it moves
and is modeled by a class \textbf{Event}.

The events are saved in a priority queue realized using heap. This was
considered one of the simplest implementations that is supposed to work good 
even for big values for the number of clients.

Other class that is used in the program is the class \textbf{RunStatistics}. It
saves information about the wait times, losses and response times. 

For the random values we use the standard python library \textbf{random}, which
is capable of giving uniformly distributed random int values in some range,
double value in the range [0, 1) or exponentually distributed values.

\subsection*{Simulator execution} % (fold)
\label{sub:Simulator execution}
  The simulation consists of the following steps:
  \begin{enumerate*}
    \item Parse the parameters and the graph topology.
    \item Execute '-repetitions' times the simulation that consists of:
      \begin{enumerate*}
        \item Generate random exponentually distributed times when the client 
          arrivals should occur, create events for them and put them in the queue.
        \item Initialize the variables for the times up until which the servers
          are busy with 0, initialize the number of clients in each queue to 0.
        \item Process for each event in the queue with regard of the type of the
          event, the server on which it is, the graph topology and its
          probabilities, the number of clients in the server queue. If the 
          event is of type 0 (arrival), then we determine the time at which it
          will be processed, update the appropriate statistics, queue sizes and
          time up until which the server is busy. If the type is 1 (departure),
          then we determine the destination to which the packet would like to 
          go and the time at which it will arrive. If the queue of the
          destination is full, the event is dropped, otherwise if the
          destination is the exit from the system, we just count the event and
          again drop it. If, however, the destination is not the exit point, 
          then we insert a new arrival event in the queue with the appropriate
          information.
      \end{enumerate*}
    \item In the end of each repetition, update the global statistics (squared
      and normal).
    \item Print the confidence intervals and output the comma-separeted-values
      (CSV) file if requested by the user.
  \end{enumerate*}

\newtheorem*{note1}{Note}
\begin{note1}
  From theoretical point of view, it is better to drop packages when they arrive
  and the queue is full, however as the communication time is supposed to be
  negligable and for that reason - ignored, the elements in the queue are not
  expected between the departure and the arrival at the next server.
\end{note1}
% subsection Simulator execution (end)

\end{document}
