\documentclass[12pt]{article}
\usepackage[utf8]{inputenc}
\usepackage[english]{babel}
\usepackage{amsfonts}
\usepackage{amsthm}
\usepackage{amsmath}
\usepackage{amscd}
\usepackage{amssymb}
\usepackage{graphics}
\usepackage{graphicx}
\usepackage{fancyhdr}
\usepackage{multicol}
\usepackage{listings}
\usepackage{xcolor}
\usepackage{enumitem}

\oddsidemargin 0mm
\evensidemargin 0mm
\topmargin 0mm
\textheight 216mm
\textwidth 165mm

\pagestyle{fancy}
\fancyhf{}
\fancyhead[CE,CO]{VAR paper evaluation by Ivaylo Petrov}
\fancyfoot[CE,CO]{\thepage}

\lstset{basicstyle=\ttfamily,
  mathescape=true,
  escapeinside=||,
  emph={Task, Operation, for, each, if, else, endif, endfor, while, then, When,
  do, repeat, until, endrepeat},
  emphstyle={\color{gray}\bfseries\itshape}}

\setlist[itemize]{noitemsep,nolistsep}
\setlist[enumerate]{noitemsep,nolistsep}

\def\CC {{\mathbb C}}        
\def\RR {{\mathbb R}}       
\def\ZZ {{\mathbb Z}}      
\def\NN {{\mathbb N}}     
\def\be  {\begin{eqnarray}} 
\def\ee  {\end{eqnarray}}  
\def\ben {\begin{eqnarray*}} 
\def\een {\end{eqnarray*}}   
\newcommand{\hr}{\rule{\linewidth}{0.1mm}}
\newcommand{\bighs}{\hspace{15pt}}
\newcommand{\hs}{\hspace{10pt}}
\renewcommand{\tilde}{\overset{-}}

\newenvironment{itemize*}{
  \begin{itemize}
    \setlength{\itemsep}{0pt}
    \setlength{\parskip}{0pt}
    \setlength{\parsep}{0pt}
}{
  \end{itemize}
}

\newenvironment{enumerate*}{
  \begin{enumerate}
    \setlength{\itemsep}{0pt}
    \setlength{\parskip}{0pt}
    \setlength{\parsep}{0pt}
}{
  \end{enumerate}
}

\newtheoremstyle{plain}{1pt}{0pt}{}{}{}{}{.5em}{\thmname{\textbf{#1}}:\thmnote{#3}}

\theoremstyle{plain}

\everymath{\displaystyle}

\renewcommand{\labelenumii}{\arabic{enumi}.\arabic{enumii}}

\begin{document}

\bibliographystyle{plain}

\section*{\centering
  Paper evaluation of\\
  \emph{"KinectFusion: Real-time 3D Reconstruction and Interaction Using a
    Moving Depth Camera"}\\
  by\\
  Ivaylo Petrov
}

\hr

The following article is going to provide information about \emph{Real-time 3D
Reconstruction and Interaction Using a Moving Depth Camera}. The main focus 
will be on the paper \emph{"KinectFusion: Real-time 3D Reconstruction and
Interaction Using a Moving Depth Camera"} by Shahram Izadi, David Kim,
Otmar Hilliges, David Molyneaux, Richard Newcombe, Pushmeet Kohli,
Jamie Shotton, Steve Hodges, Dustin Freeman, Andrew Davison, Andrew Fitzgibbon.

\section{Abstract} % (fold)
\label{sec:Abstract}
  Creation of an relatively accurate 3D map of some objects with a low-cost 
  hand-held device has been a goal of a number of researches. Now with a Kinect
  camera and appropriate software that is possible. Furthermore, with this 
  camera now we are not only able to rapidly create a detailed 3D reconstruction
  of a limited size scene (a room for example) in real time \cite{kinectfusion},
  but we can also use this reconstruction to create geometry-aware augmented
  reality with physical interactions within it. 

  KinectFusion is a software that is capable of using a Kinect camera to do all
  that without the need of additional widely used devices such as trackers. It
  also provides us with the possibility to enable real-time multi-touch
  interactions anywhere, allowing any planar or non-planar reconstructed
  physical surface to be appropriated for touch.
% section Abstract (end)

\section{Introduction} % (fold)
\label{sec:Introduction}
  For quite some time people have been fascinated by the possibility with an 
  affordable, hand-held device and in real time, to scan accurately 3D shapes,  
  to create geometry-aware augmented reality, and to create physically based
  interactions within it. So far, however, this has been impossible. With the
  inventing of the Kinect camera the world is presented with the possibility to
  create relatively accurate 3D maps for very low price. For that reason this
  device rapidly gained popularity among researchers and enthusiasts alike.

  While this device does not provide a conceptually new approach, structured
  light technique to obtain depth map is not a new idea, its low-cost and
  relative accuracy contribute greatly to its success. As the authors of the
  article note, however, the noise that is caused by fluctuations and holes
  where reads were not obtained need to be taken care of with some algorithms in
  order to make the device really useful for the above scenarios. 

  The already described applications of this device, with the appropriate
  software, range from gaming and physics to CAD programs. For all those things,
  however, an accurate model should be inferred from the noisy point-based data.
  To do that the authors consider different approaches and finally decide to use 
  \textbf{volumetric surface representation}. Although this approach has some 
  drawbacks, i.e. hight memory consumption, which might potentially lead to high
  computational cost (as noted in
  \cite{representaion-and-rendering-of-implicit-surfaces}). There however is a
  solution to this problem, that the authors suggest - the use of octree.
  Although there is no additional information about this approach in the
  original article, it will be explained in the appropriate part of this paper. 

  As the system is designed in a way that allows moving of the Kinect camera and
  interaction of users with the scene, the authors needed to solve a number of
  difficulties that those two characteristics introduced. In order to track the
  camera position, they have decided to track the \emph{6 degrees-of-freedom
  (DOF)}. They divide the input in two parts - foreground and background and use
  only the data from the background to calculate the 6DOF, because as algorithms
  generally depend on the staticness of the scene, it is impossible to
  accurately calculate the camera position if all the data is considered. This
  allows complex user interactions with the camera that does not significantly
  degrade the quality of the camera tracking. This division of foreground and
  background objects makes possible the use of the system to detect multi-touch
  interactions with physical surfaces of any type - planar or not.

% section Introduction (end)

\section{Related work} % (fold)
\label{sec:Related work}
  Although scene geometry reconstruction using various techniques such as active
  sensors, passive cameras, online images or from unordered 3D points, is well
  studied in the field of computer graphics and vision, the possibility of
  interaction with the scene is what distinguishes \cite{kinectfusion} from all
  proposed solutions until now. The area of Simultaneous Localization and
  Mapping (SLAM) is also tightly connected with the proposed system, as it has
  to determine its position on in real time as new observations are provided,
  while at the same time create a precise 3D representation of the observed
  objects and cope with user interactions. How this problem is solved in the
  described article will be explained later.

  Other differences between KinectFusion and previous works can be described as:
  \begin{itemize}
    \item \textbf{Interactive rates} In contrast with most of the already
      existing solutions, KinectFusion wants to provide real-time computation of
      both camera position tracking and world geometry reconstruction. This is
      essential for some of the applications that are targeted by the authors
      and most of all for games and augmented reality, where direct feedback and
      user interactions are key to the usability of the system. By the time that
      the article was published, there was no system, to the knowledge of the
      authors, that provided interactive rates for those two things.
    \item \textbf{No explicit feature detection} Unlike some of the previously
      created techniques that use Structure from Motion (SfM) or RGB plus depth
      (RGBD) detect sparse scene features, the proposed solution uses directly
      in some way all the data acquired by the Kinect sensor. This also makes
      the system independent of the light conditions in the observed scene.
    \item \textbf{High-quality geometry reconstruction} The authors of the
      system, unlike many other before them, do not want to sacrifice details in
      the reconstructed model in order to obtain real time tracking. Often,
      sparse maps are used for the localization, but they can not be used as a
      mean to reconstruct the observed geometry. Instead of that, the authors
      use volumetric surface representation, which enables them to directly
      reconstruct surfaces that can also be used for rendering of the scene,
      possibly with objects from augmented reality, in real time.
    \item \textbf{User interaction assumed} In most of the previously provided
      studies, the scene was expected to be completely static in order for the
      systems to work. The authors of \cite{kinectfusion} created a system that
      along with tracking the camera and reconstructing the scene, allow user
      interaction. This requires the selection of an representation, that can
      easily deal with this dynamicity, without compromising the real-time
      nature of the computations.
    \item \textbf{Mobility without additional infrastructure} The idea of the
      project is to create mobile (at least to some degree) system that can be
      used to reconstruct any indoor spaces. This distinguishes the proposed
      solution from many earlier works that used fixed or large sensors, or
      sensors that are fully embedded in the environment. Also in order the
      application to easier to use, the authors wanted to avoid using
      augmentation of the space and the interacting entities with some markers.
    \item \textbf{Size scale} Additional requirement that distinguishes the
      proposed approach is the support of bigger scenes - a whole room for
      example. In the past most of the detailed reconstruction techniques that
      were proposed would be usable only for relatively small geometries.
    \item \textbf{Low-cost scanning} The combination of low-cost, real-time and
      excellent geometry details has not been shown by any of the previous
      works. Those things contribute to much better user experience, as the user
      can instantly verify that the scanned object has the necessary properties
      and proceed with the use of the created model. Furthermore, the system can
      be used not only to scan static scenes, but also to scan objects that are
      rotated by the user in front of the camera. For the latter, however, there
      are a few requirements that should be satisfied in order this use to be
      applicable. First of all, the object should occupy a sufficiently large
      portion of the depth map in order ICP to be able to converge correctly.
      The moving of the object should not be too quick, as this might cause
      problems to the tracking algorithm as well.

      It is still possible to scan smaller object, provided that first the
      system is able to reconstruct sufficiently accurate representation of the
      surrounding world and just after that the object is presented for
      scanning.
  \end{itemize}
% section Related work (end)

\section{KinectFusion} % (fold)
\label{sec:KinectFusion}
  Following the example of the example of the authors of \cite{kinectfusion}, we
  will now present a number of applications of the KinectFusion system and after
  that some of its parts will be described in more details and additional
  information will be provided from
  \cite{representaion-and-rendering-of-implicit-surfaces} and \cite{icp}.

  The system tries to constantly keep track of the camera position in the space,
  calculating constantly the 6DOF and adding new data as camera is moved or new
  observations are added for places, where before that there were holes, i.e.
  there was no information for that part of the scene. The camera potion does
  not have to be too big in order new data to be acquired, this can happen, for
  example, because of a slight camera shake. The authors point out that given
  the real-time speed of the reconstruction, the level of details of the
  reconstructed objects is more then good. Using the RGB camera that the Kinect
  sensor has, it is also possible to map textures to the reconstructed objects
  as shown in Figure~\ref{fig:model_with_texture}B.

  \begin{figure}[h]
    \centering
    \caption{Model with texture mapped from Kinect RGB camera}
    \includegraphics[height=40mm]{graphics/augented_reality.png}
    \includegraphics[height=40mm]{graphics/model_with_texture.png}
    \includegraphics[height=40mm]{graphics/augmented_reality_with_a_number_of_small_spheres.png}
    \label{fig:model_with_texture}
  \end{figure}

  One of the possible application of the system is for \textbf{creation of
  augmented reality}. It is possible to add some 3D objects in the model,
  created by the system and then during the rendering to show them to the user
  as viewed from the position of the kinect camera, reflecting or casting
  shadows to neighboring objects. The requirement to view the scene from the
  kinect camera position comes from the fact that for texture mapping, the
  system uses the build-in Kinect RGB camera, which is at the same position as
  the depth camera.  In Figure~\ref{fig:model_with_texture}A an additional metal
  sphere can be seen, that is not presented in the physical scene, but that
  looks exactly as if it was at that position during the taking of the picture.

  Another application is \textbf{physical world dynamics simulation}. That means
  that, due to some properties of the implementation that preserve some
  information about the velocity of the objects in the scene, it is possible to
  simulate rigid body collisions between real and virtual objects that are
  physically precise and, of course, real-time. This can have great application
  in areas such as gaming and robotics. The number of virtual objects is not
  precisely limited, it is believed that as much as a few thousands of them can
  be simultaneously introduced in the scene, without breaking the time bounds of
  the execution. Such scene can be viewed in
  Figure~\ref{fig:model_with_texture}C, where there is a big number of small
  spheres that are added to the real scene.
% section KinectFusion (end)

\begin{thebibliography}{widest entry}
  \bibitem{kinectfusion} KinectFusion: Real-time 3D Reconstruction and
    Interaction Using a Moving Depth Camera, Shahram Izadi et al., 2011
  \bibitem{representaion-and-rendering-of-implicit-surfaces} Representation and
    Rendering of Implicit Surfaces, Christian Sigg, 2006
  \bibitem{icp} P. J. Besl and N. D. McKay. A method for registration of 3D
    shapes. IEEE Trans. Pattern Anal. Mach. Intell., 14:239–256, February 1992. 
\end{thebibliography}

\hfill\\
\hfill\\
\hfill\\
\hfill\\
\hfill\\
\hfill\\
\hfill\\

The proposed simulation engine is entirely written in Python. It is tested with
python version 3.3, however it should work with older version as well.

The simulation engine can receive a number of parameters: 
\begin{itemize*}
  \item '-nc' $\rightarrow$ how many clients should be provided as input.
  \item '-queue-sizes' $\rightarrow$ a comma-separated list of the sizes for
    each of the queues. For the queues that are infinitely big, the user has to
    enter a value equal or greater than the number of clients.
  \item '-muis' $\rightarrow$ a comma-separated list of the speed for
    processing the clients for each server.
  \item '-lambdas' $ \rightarrow $ a comma-separated list of the speed at which
    the clients arrive to the input servers. Please note that the code was
    never tested with more than one input - server 1.
  \item '-graph' $ \rightarrow $ The file that describes the graph topology.
    That file contains the number of servers, input servers, output servers,
    and connections between servers with the probability to follow some of 
    those connection. For example is we have 0 2 0.5 this means that from 
    server 1 we can go to server 3 and the probability to go there, and not to
    some of the other servers that 0 is connected to, is 0.5.
  \item '-output-file' $ \rightarrow $ A file to which to append a
    comma-separated representation of the output (useful for processing with
    excel). It is optional.  
  \item '-repetitions' $ \rightarrow $ The number of execution of the simulation
    from which the mean statistics are extracted. If it is not provided, 100 is
    used.
\end{itemize*}

Those parameters are saved in an instance of the class \textbf{Configuration}
that is passed to the methods that need it.

For the purpose of this section when we say \textbf{event} we will refer to the
event of client packet being received by some of the servers or departing from
the servers.

The main idea of the simulation engine is to have a set of events and process 
them according to the time at which they occur. Each event has time when it
arises, type - arrival at server or departure, server to or from which it moves
and is modeled by a class \textbf{Event}.

The events are saved in a priority queue realized using heap. This was
considered one of the simplest implementations that is supposed to work good 
even for big values for the number of clients.

Other class that is used in the program is the class \textbf{RunStatistics}. It
saves information about the wait times, losses and response times. 

For the random values we use the standard python library \textbf{random}, which
is capable of giving uniformly distributed random int values in some range,
double value in the range [0, 1) or exponentually distributed values.

\subsection*{Simulator execution} % (fold)
\label{sub:Simulator execution}
  The simulation consists of the following steps:
  \begin{enumerate*}
    \item Parse the parameters and the graph topology.
    \item Execute '-repetitions' times the simulation that consists of:
      \begin{enumerate*}
        \item Generate random exponentually distributed times when the client 
          arrivals should occur, create events for them and put them in the
          queue.
        \item Initialize the variables for the times up until which the servers
          are busy with 0, initialize the number of clients in each queue to 0.
        \item Process for each event in the queue with regard of the type of the
          event, the server on which it is, the graph topology and its
          probabilities, the number of clients in the server queue. If the 
          event is of type 0 (arrival), then we determine the time at which it
          will be processed, update the appropriate statistics, queue sizes and
          time up until which the server is busy. If the type is 1 (departure),
          then we determine the destination to which the packet would like to 
          go and the time at which it will arrive. If the queue of the
          destination is full, the event is dropped, otherwise if the
          destination is the exit from the system, we just count the event and
          again drop it. If, however, the destination is not the exit point, 
          then we insert a new arrival event in the queue with the appropriate
          information.
      \end{enumerate*}
    \item In the end of each repetition, update the global statistics (squared
      and normal).
    \item Print the confidence intervals and output the comma-separeted-values
      (CSV) file if requested by the user.
  \end{enumerate*}

\newtheorem*{note1}{Note}
\begin{note1}
  From theoretical point of view, it is better to drop packages when they arrive
  and the queue is full, however as the communication time is supposed to be
  negligable and for that reason - ignored, the elements in the queue are not
  expected between the departure and the arrival at the next server.
\end{note1}
% subsection Simulator execution (end)

\end{document}
