\documentclass[11pt]{article}
\usepackage{fullpage}
\usepackage{graphicx}
\usepackage{subfigure}
\usepackage{color}
\usepackage{hyperref}

\definecolor{dark-gray}{gray}{0.20}

\hypersetup{
  colorlinks=true, % set true if you want colored links
  citecolor=dark-gray, % chooses some color if you want the citations to stand out
  filecolor=black, % chooses some color if you want files to stand out
  linktoc=all, % set to all if you want both sections and subsections linked
  linkcolor=black, % choose some color if you want links to stand out
  urlcolor=darkgray,% chooses some color if you want the urls to stand out
}

\newcommand{\HRule}{\rule{\linewidth}{0.5mm}}

%numeroter les pages
\pagestyle{plain}


\begin{document}

\begin{titlepage}

\begin{center}

% Upper part of the page

 

\includegraphics[width=0.8\textwidth]{./header}\\[1cm]
\textsc{\Large Master research Internship}
\vspace{1cm}

\includegraphics[width=0.11\textwidth]{./logo_ENIB}
\includegraphics[width=0.11\textwidth]{./ens-cachan_bretagne}
\includegraphics[width=0.11\textwidth]{./esir}
\includegraphics[width=0.11\textwidth]{./enssat}
\includegraphics[width=0.11\textwidth]{./insa-rennes}
\includegraphics[width=0.11\textwidth]{./rennes1}
\includegraphics[width=0.11\textwidth]{./supelec}
\includegraphics[width=0.11\textwidth]{./logoUbs}
\includegraphics[width=0.11\textwidth]{./UBO}
\includegraphics[width=0.11\textwidth]{./tel-br}


  
\vspace{1cm}
\textsc{\Large Bibliographic report }\\[0.5cm]


% The title of your report
\HRule \\[0.4cm]
{ \Large \bfseries Combinatorial Optimization for Fast Scaffolding }\\[0.4cm]

\HRule \\[1.5cm]

% Author and supervisor(s)
\begin{minipage}{0.4\textwidth}
\begin{flushleft} \large
\emph{Author:}\\
Ivaylo \textsc{Petrov}
\end{flushleft}
\end{minipage}
\begin{minipage}{0.4\textwidth}
\begin{flushright} \large
\emph{Supervisor:} \\
%
% name(s) of your supervisor(s)
Rumen \textsc{Andonov} \\
Antonio \textsc{Mucherino} \\
% Name of your team
GenScale
\end{flushright}
\end{minipage}

\vfill


% INCLUDE HERE THE LOGO OF YOUR INSTITUTION
%\textbf{INSERT ``\%'' IN FRONT OF ALL THE LOGO YOU DO NOT NEED - A SINGLE ONE SHOULD REMAIN AT THE BOTTOM OF THIS PAGE}
%\begin{flushleft}
%\includegraphics[width=0.15\textwidth]{./supelec}\\
%\end{flushleft}
%\begin{flushleft}
%\includegraphics[width=0.15\textwidth]{./logoUbs}
%\end{flushleft}
%\begin{flushleft}
%\includegraphics[width=0.15\textwidth]{./UBO}
%\end{flushleft}
%\begin{flushleft}
%\includegraphics[width=0.15\textwidth]{./tel-br}
%\end{flushleft}
\begin{flushleft}
\includegraphics[width=0.15\textwidth]{./rennes1}
\end{flushleft}
%\begin{flushleft}
%\includegraphics[width=0.15\textwidth]{./insa-rennes}
%\end{flushleft}
%\begin{flushleft}
%\includegraphics[width=0.15\textwidth]{./esir}
%\end{flushleft}
%\begin{flushleft}
%\includegraphics[width=0.15\textwidth]{./enssat}
%\end{flushleft}
%\begin{flushleft}
%\includegraphics[width=0.15\textwidth]{./ens-cachan_bretagne}
%\end{flushleft}
%\begin{flushleft}
%\includegraphics[width=0.15\textwidth]{./logo_ENIB}
%\end{flushleft}
\end{center}
\end{titlepage}



%************************************************************%

\begin{abstract}
  \textcolor{red}{Here should appear your abstract - Should be fifteen lines
  long }
\end{abstract}

% compile twice to get the table of contents
\tableofcontents
\newpage

\setcounter{page}{1}

%*****************************************************************%
\section{Introduction}
\label{sec:Introduction}

\subsection{Context}
\label{sub:Contex}
For a very long time some people have been fascinated with the diversity of
living beings on the planet. They have spent their lives trying to understand
the differences and similarities between species. They have observed various
characteristics of each specie. With the advance of technology, more and more
characteristics have been documented for every living organism, often each new
one with a smaller size than the previous. Over time scientists have discovered
that inside each living organism there is a large number of building units, that
they called cells. Inside each cell there is a molecule called DNA that can be
depicted as a sequence of nucleotides. It turned out that this big molecule
encodes the genetic instructions used in the development and the behavior of all
known living organisms and many viruses. In order to better understand the
development and behavior of every living being, scientist have to better
understand this molecule - how to decipher it. This, however, is not an
easy task. There are a number of problems that need to be solved before the
deciphering can be performed.

The first problem is that the structure of this molecule is specific for each
individual. This means that it is not the same for any
two persons, for example, except if they are identical twins. This makes the
process of determining the structure of this molecule very hard. Scientists want
to study it, nevertheless, and for that reason the DNA of some individuals
should be sequenced, meaning that the sequence of more basic building
blocks\footnote{Those building blocks are called nucleotides} inside the DNA
should be found.

Another problem is after the structure is obtained, how to understand what it
actually means - how to decipher it.

This is somewhat similar to a situation where a compiled version of some program
in an unknown (lets say alien) programming language is obtained. In order to
study it, scientist would greatly benefit from having the source code of the
program, even though it will not be understandable for them. They will be able
to find similarities in the code that should lead to similarities in the program
execution.

It is the same with the DNA structure. Scientists will be able to understand
some of the functions of parts of the DNA molecule on the basis of similarities 
with some of its other parts.

As it was already noted, the obtaining of the structure of the DNA is not an
easy task. The problem is that there is no known technology that can read the
structure of the entire DNA molecule in one read. For that reason, a number of
technologies that provide shorter reads have emerged. The difference between
them is not only the size of the reads that they provide, but also the price for
each DNA sequencing and the way the reading positions are chosen. For newer technologies
the reading positions are chosen at random, which leads to a need of high coverate\footnote{
How many times the sum of the lengths of the reads is greater than the size of the 
genome itself} in order all the parts of the genome to have good chances of being represented.
As for the difference in the read sizes, Sanger sequencing reads are of sizes between
500 and 1000 bases, which is quite smaller than the size of the human genome which
is around 3,234.83 mega bases, but it is still much bigger then that of the
newer technologies that provide reads with sizes from 20 to 100 bases. The
difference in the price and the throughput, however, makes the newer technologies
much more preferable than the old ones.

This, nevertheless, puts a new challenge for scientists to overcome: how to gather
all those small pieces together. There are two ways to approach this problem.
The first is, if there is already some reference sequence (it does not matter
how it was obtained - it can be through some very costly technology). In that
case it is possible to try to put the pieces on the template, allowing some
small differences. Although this approach is simple, it does not allow the scientists to
obtain a genome that is very different from the reference one. Furthermore, the
choice of the reference genome, in the case where there is more than one,
affects the result, which means that this approach is not always very accurate.
For those reasons, a second approach was proposed. It consists in assembling
assemble the pieces together based on some overlappings that they share or some
other additional information about them. This approach is called de novo genome
assembly.

A naive way to approach the problem of whole-genome shotgun (WGS) de novo
assembly\footnote{This is the so far described process of assembling small reads
from different parts of the genome, usually with very high coverage that tries
to compensate the randomness of the sequenced places} is by trying each piece
with every other and see if they match well together. After that use some
heuristics to determine which of the matching ones should be brought together.
Even though this approach was not efficient, it was used for some time. In 2001
Pavzener et al \cite{pevzner-et-all-2001} proposed an approach that uses a structure
called de Bruijn graph in order to make the process efficient. More information
about this approach will be provided below.

As the short reads were not sufficient to represent more complex parts of the
genome, including some partial repetitions, i.e. long enough parts that occure at
multiple positions in the genome and that are longer than the read size, some additional information was
added to the output of the sequencing of the new technologies - pair reads. This
is information about relative orientation and distance between two reads. With
this new information the DNA structure can be represented more correctly,
allowing the creation of better tools to assemble genomes.

% subsection Context (end)

\subsection{Assembling strategies}
\label{sub:Assembling strategies}
The process of assembling the small reads of DNA into bigger parts can be
divided into at least two phases. The firs one creates contigs from the read,
most commonly using some overlapping criteria. \textbf{Contigs} are gapless
sequences of nucleotides. The second phase is the \textbf{scaffolding} or the
phase where contigs are oriented and ordered linearly in such a way that
minimizes the number of unsatisfied scaffolding constraints (as defined in
\cite{grass}). Scaffolding constraints can be relative position and relative
orientation of pairs of contigs that can be inferred from the mate-pairs, for
example.
% subsection Assembling strategies (end)

\subsection{Scaffolding}
\label{sub:Scaffolding}
As the scaffolding is the last step in the genome assembly process, it is one of
the most important ones. The problem, however, is very complicated and it is proven 
to be NP-hard both when only the contig orientation is taken into account and  
when the orientation is known and the ordering is to be determined. 
% subsection Scaffolding (end)

The focus article will be on the scaffolding problem, however some details about
process that are necessary for a successful scaffolder to operate will also be
provided. In the next section an overview of the currently available scaffolding
algorithms will be given with a substantial level of details. After that some
additional steps that can greatly improve the results from the scaffolding will
be shortly described. Next, a description on how the performance of a scaffolder
can be measured will be provided. In the end the idea of the internship will be
shortly presented.

% section Introduction (end)

%*****************************************************************%
\section{Related word} % (fold)
\label{sec:Related word}

\subsection{Classification of the available scaffolders} % (fold)
\label{sub:Classification of the available scaffolders}
Currently there are a number of systems that could put contigs into scaffolds.
Some of them have been created more than 10 years ago, when the computers have
had very restricted amount of random access memory, slower processors than
those of the computers today, and most of all, very few successful utilizations
of parallel programming techniques. This is why most of the earliest efforts in
solving this problem tried to come up with very fast, most of the time linear
algorithms in the size of the graph that they analyzed. The speed, however, was
at the expense of accuracy.

Having said all that, the algorithms can be divided in two main groups: those
that use some form of \emph{de Bruijn graph} (\cite{pevzner} and
\cite{velvet-scaffolding}) and those that use some graph G(V, E), for which the
nodes V are the contigs and the edges are some scaffolding constraints (there
are different names for those graphs, including \emph{contig graph},
\emph{scaffolding graph}, etc).

Another classification can be by the approach they have to the use of
heuristics. Some systems directly use heuristics (\cite{greedy-path-merging},
\cite{SOAPdenovo}, \cite{SSPACE}), using the graph structure very superficially,
while others try to find some patterns or properties in the graph that will
allow them to solve the problem exactly for some parts of the graph, while using
heuristics to solve only the remaining, more complex parts (\cite{SOPRA},
\cite{MIR}, \cite{Opera}, \cite{SCARPA} and \cite{grass}).

Although there is no strict classification of the different systems, there are
clear similarities between some of them, which will facilitate their
understanding and for that reason will be used in the rest of the section.
% subsection Classification of the available scaffolders (end)

\subsection{Scaffolders that use de Bruijn graph} % (fold)
\label{sub:Scaffolders that use de Bruijn graph}
Those methods were created as an extension of the use of de Bruijn graphs for
the contig assembly. Before \cite{pevzner} all the approaches for contig
assembly used different heuristics that would lead to as much as 19\%
misassembled contigs. Then the use of de Bruijn graphs was proposed with
the problem being formulated as finding of eulerian path in that graph. Although
the proposed system by \cite{pevzner} - EULER-SF uses pair reads for the genome
assembly, it does not directly create scaffolds and thus it might not be
considered as a scaffolder. In \cite{SOPRA} however, the authors point it as a
scaffolder that uses pair read information to improve the walk in the de Bruijn
graph. For that reason a short description of the algorithm will be provided.

\subsubsection{EULER-DB} % (fold)
\label{ssub:EULER-DB}
This approach as described by \cite{pevzner} focus on creating scaffolds from
long and reliable contigs, rather than from all the contigs. This greatly
reduces the difficulty of the problem. The basic idea of the algorithm is to try
to "explain" as many pair-reads as possible by adding them between a source node
and a sink node in the de Bruijn graph. There are some techniques that are used
in order to lower the impact of chimeric contigs or errors in the mate-pairs.
For more details see \cite{pevzner}.

% TODO: Try to explain a little bit better and in more details.
% subsubsection EULER-DB (end)

\subsubsection{Velvet} % (fold)
\label{ssub:Velvet}
The most famous of the de Bruijn graph based scaffolders is Velvet
\cite{velvet-scaffolding}. The idea of this system is to try to connect all the
unique nodes (the data from the contig assembler suggests that those contigs
should be presented in the scaffolds only once). First, the unique nodes are
determined using statistical approach that verifies the coverage of the contig.
After that a primary and secondary scaffolds are formed in the following manner:
for the primary scaffold a distance between each pair of nodes is estimated
using the maximum likelihood estimator. The complete set of inter-node distances
are called primary scaffold. For the secondary scaffold the system proceeds as
follows: for a unique node A, the distances to all the neighbours of its primary
neighbours (neighbours in the primary scaffold) are calculated. This is trivial
as the sign of the distance signifies at which side of the other node the node
is located and primary estimations are already available. It is nevertheless
good to note that the secondary estimations might not be correct, as they
involve contigs that are not unique.

Unique nodes are after that connected using a heuristic depth-first search that
takes into account the distance estimations and tries to reach the closes unique
node from its starting one. This way unique nodes are connected through a number
of iterations of the so far described approach.

It is important to note that on the general the heuristic depth-first search can
go in an infinite loop. Velvet takes care so that this does not happen.

Using the paths between the unique nodes, the scaffolds are formed.

% TODO: Try to find a better description of the graph and present it.
%\textcolor{red}{The graph is not well defined I think - it should be de Brujin
%graph with read pair information}
% subsubsection Velvet (end)

% subsection Scaffolders that use de Bruijn graph (end)

\subsection{Scaffolders that use scaffolding constraints graphs} % (fold)
\label{sub:Scaffolders that use scaffolding constraints graphs}
As it was previously noted, the typical characteristics of all the following
systems for contig scaffolding is the fact that they use a graph where the nodes
of the graphs are the contigs and the edges of the graph are constraints for the
scaffolding. All of them try to resolve a similar problem, although for some of
them it is less generic. The problem can be formulated as follows: given a set
of possibly incompatible scaffolding constraints, find the "smallest" subset of
them that has to be removed in order the remaining constraints to be compatible.
The quotation marks on smallest are placed as some of the constraints can depend
on one another and a simple counting might not be optimal. For example the
removal of links that point to a \textbf{chimeric contig} (a contig that has
parts from two distinct parts of the genome in it) should not be considered as
equivalently significant as removing a link from a normal contig, as otherwise
the true nature of the problem will not be adequately modeled.

The above definition of the problem is taken from \cite{grass} as it was more
precise than the other definitions that only take into account the constraints
imposed directly by the mate-pairs. This allows the authors of \cite{grass} to
use additional constraints such as similarity to a reference genome and others
in addition to the mate-pair constraints, in order to obtain a better scaffold.

\subsubsection{Greedy Path-Merging} % (fold)
\label{ssub:Greedy Path-Merging}
The first algorithm that will be presented uses heuristic approach, without
trying to solve any part of the problem exactly. It is described in more details
in \cite{greedy-path-merging}. The algorithm operates on a graph that is
described in the following way: all contigs are mapped to two nodes in the
graph. Between those two nodes there is an edge, named contig-edge, with length,
the length of the contig. One of the nodes is labeled with label \emph{s} and
the other with label \emph{t}.

For each mate-pair that connects two distinct contigs we add an edge in the
graph, given that this is the only edge that connects those two contigs (and that
it does not repeat information that can be obtained from a number of other
edges). Otherwise a statistical approach is used to bundle the mate-pairs that
have similar length (for more information see \cite{greedy-path-merging}). Such
edges are called mate-edges.

The graph obtained in the already described way is called contig-mate-pair
graph. For it we execute the following algorithm. First, all contig-edges are
said to be a part of the scaffolding (they are selected). This provides a number
of very short paths of selected edges in the graph. For every mate-edge we try
to combine the paths it connects. When there is no significant overlapping
between the two paths, it is possible to combine them. In that case we check if
the number of edges that will be \emph{happy}, meaning that the contigs that
they connect are oriented in accordance to the relative ordering imposed by the
edge and that the distance between them is close to the expected one, will have
increased more than the number of unhappy edges. Unhappy edges are those edges
that are not happy. If both conditions are true, then we connect the two paths,
setting the edges selections accordingly.

After all the meta-edges are used, there is one additional step which tries to
fill any gaps in the scaffold using some of the unused contigs and mate-pairs.

Although the presented algorithm uses some heuristics that can potentially lead
to a solution that is not very similar to the optimal, one of its best
characteristics is its complexity, which is $O(m n + m ^ 2)$, where \emph{n} is
the number of contigs and \emph{m} is the number of mate-pairs. As this
algorithm was proposed in 2002 and the computational power of the available
computers was quite less than that of the computers today, it was essential to
come up with algorithms that would run in an acceptable amount of time, even if
they are not very precise, as biologists could, nevertheless, use the results
provided by such programs.
% subsubsection Greedy Path-Merging (end)

\subsubsection{SOAPdenovo} % (fold)
\label{ssub:SOAPdenovo}
The next system to be presented \cite{SOAPdenovo} was created in 2009 and at
that somewhat distant time it was aiming at assembling human genomes. This goal
implied a very fast runtime, which in turn implied the use of sufficiently
simple heuristics, that would be able to run in a reasonable amount of time. The
proposed system behaves as follows: first it maps the mate-pairs to the contigs
(note that originally we have the mate-pairs that connects two reads, not
contigs). After that a contig linkage graph is build. In it nodes are contigs
and edges are formed by mate-pairs. Edges are constructed by estimating the gaps
between the contigs, assuming that those gaps lengths follow a Normal
distribution, centered approximately at the targeted insert size. The weight of
each edge is determined by the number of read pairs that connect the two
contigs. Edge is not added if there are less than three mate-pairs that support
it.

For that graph the following steps are performed: linearization, which is the
replacement of compatible transitive lineages among a group of contigs with just
one node for which internal gaps are carefully estimated; masking of nodes that
can not be ordered due to conflicting suggestions by different edges.

As it can be seen the approach of this system is very simple, however, as it is
uses multiple libraries, potentially with different insert sizes, it hopes that
that vast amount of information that it will be able to process will compensate
the lack of sophisticated use of that information.

One of the contributions of this article is that it shows that human genome
assembly is possible even with the computers of that time, without the use of
some complicated parallelization technique. Another one is that, at least to the
best knowledge of the authors of the current article, it is the first system
that tries to use information from multiple data sources in order to resolve the
problem.

An additional contribution of this article is the effort to use disregarded
mate-pairs for filling some of the gaps that are created during the scaffolding
process. Even though the proposed gap closing technique is very naive, it still
improved to some extend the results obtained by the system.
% subsubsection SOAPdenovo (end)

\subsubsection{SOPRA} % (fold)
\label{ssub:SOPRA}
A article that really had a great influence on every following one, introducing
a number of new ideas to the previous one, was \cite{SOPRA}. In this paper the
SOPRA system was described, which would try, at least for some parts, to solve
the contig orientation problem exactly. This means that if the contig graph has
some nice properties, for some of its sub graphs the problem will be solved in a
way that will ensure the discovery of the optimal solution, in the terms of the
defined problem. Here are the details.

First the contigs are produced using some of the already existing systems. After
that the data is filtered. One of the filtering criteria is based on the
coverage. The authors have decided that they do not want to allow contigs that
are part of repeats in the scaffolding, as they would normally have mate-pairs
that would suggest inconsistent orientation. There is a higher chance that such
contigs are chimeric, which also suggest that they should be removed.

Another criteria for mate-pair removal is the length that they suggest. The
mean insert size can be estimated using contigs that are longer than it. In such
areas precise information for the insert size can be obtained by measuring the
distance between the reads in the contig. Mate-pairs that have insert length
that deviates with more than 20\% of the mean insert size, are considered
incorrect and are thus filtered.

After the data is filtered, a contig-mate-pair graph is build that uses contigs
as nodes and mate-pairs as edges. After that, the system proceeds by orienting
the contigs. Then, it orders them and in the end it detects and resolves
overlapping that has been caused by the ordering. Here is how the ordering
works: partition the graph into parts based on articulation
vertices\footnote{Those are vertices that when removed, the graph is divided in
two components}. As for most contigs there is a limited number of neighbours in
the graph, this partitioning is possible, especially for less complex genomes.
Contigs that are longer than the insert size, for example, are guaranteed to be
articulation vertexes. We solve the problem for the components separately as
there are no edges between them, thus only the articulation vertex orientation
can have impact on the ordering of the contigs. Note, however, that if any two
components suggest different orientation of the articulation node, then the
orientation of every contig in one of the connected components can be inversed,
which will lead to a correct solution.

The partitioning process is continued for each of the subgraphs until every
subgraph is non-reducible. Then the orientation problem is solved using either a
dynamic programming technique or Simulated Annealing method\footnote{A variant
of Monte Carlo method}, depending on the component size. If the graph component
has relatively small size, then the dynamic programming approach is used to
solve its orientation exactly, minimizing the number of violated mate-pairs. If
the component size is too large, however, the dynamic programming approach could
not be used, and as a consequence the authors have decided to use a Simulated
Annealing method. For more information about the those two approaches, see
\cite{SOPRA}.

The ordering of the contigs can be represented as minimizing the difference
between the expected length of each mate-pair and its actual length in the
ordering. For that step only mate-pairs that are consistent with the orientation
are used, which simplifies the problem of ordering. If after solving this
problem there are mate-pairs for which their lengths are significantly different
from the expected one, the contigs that are attached to them are removed and the
contig scaffolding is restarted on the remaining contigs. The reason why those
contigs are removed is that there is higher change for contig misassembly for
some of them and as the system tries to greatly improve the accuracy of the
assembly, not so much the length of the result, it chooses the safer approach.

This process is repeated until all the mate-pair lengths are close to the
observed in the scaffolding. Then an overlapping detection step is executed. It
tries to detect contig overlapping, based on a 'density profile', which is a
quantity that represents how many contigs cover each region of a scaffold. After
that contigs that cause the overlapping are removed. As it is not possible to
detect exactly which contigs caused the problem, if the number of the ones that
might be to blame is low, the authors propose all combinations to be verified
and the one with the smallest number of deleted contigs to be chosen. If the
number of contigs that could potentially cause the problem is too high, then
random subsets of contigs are tested and a score is given to all of them. The
score is determined by the density of the scaffold and the production of many
small scaffolds as a result of the removal. The subset with the best score is
chosen in the end to be removed.

In the previous paragraph we mentioned that only the contigs that might cause
the problem are checked, but we did not explained how they are detected. It is
done using Simulated Annealing method and contig labeling. See \cite{SOPRA} for
more details how this is done.

As it was previously pointed out, this article presents a set of new ideas that
are largely used in the articles after it. If we have to limit ourselves to
pointing out just the major improvements, they are the division of the graph in
partitions that are handled separately and the use of exact techniques for
solving some parts of the scaffolding problem. It is no surprise that this
system provided excellent quality scaffolds in terms of low error rates while in
the same time keeping good length of the scaffold.
% subsubsection SOPRA (end)

\subsubsection{MIR} % (fold)
\label{ssub:MIR}
Another system that tries to bring the idea of exact solving of the scaffolding
problem even one step further, is MIR introduced in \cite{MIR}. The idea of this
method is similar to some extend to the one presented in the previous section.
Once again we first filter the data and then build a graph with vertices that
represent contigs, however in this case the edges are sets of mate-pairs that
indicate the same orientation. Each edge has \emph{support} which is the number
of mate-pairs that are bundled in this edge. They also have lengths that are
calculated from the lengths indicated by the mate-pairs. This reduction of the
size of the graph, at least in terms of edges, is supposed to improve both the
separation of the graph into components, which is again used, and also the
runtime, as often the runtime depends on the number of edges in that graph.

In order to exactly solve each subproblem, the authors have decided to limit the
size of each subcomponent. This is achieved using a heuristic, which removes the
edges with lower support, that would cause too big components. This is done by
sorting the edges by their support and then adding each of them to the graph.
Using the technique of \cite{westbrook-tarjan}, which enables the authors to
keep track of the biconnected components dynamically as new edges are added to
the graph, it is decided whether to add the edge, or not, based on the resulting
graph structure.

After that for each graph component an minimization problem is solved using
mixed integer programming, that tries to minimize the sum of the support of the
remove edges. See \cite{MIR} for more details on the optimization.

The innovations of this article can mainly be summarized as the bundling of the
edges together, the attempt to solve the scaffolding exactly, after some
reduction of the constraints of the problem and the fact that it tries to both
orient and order the contigs exactly in the scaffolding. Another noticeable
characteristic that it has is that it is quite faster than SOPRA.

The results of this technique are very good, although it does not accomplish the
precision of SOPRA during the assembly, it still provides very close error
rates, while providing very smaller number of scaffolds. It is not clear,
nevertheless, whether longer scaffolds are more important than error rates and
how to manage the two. Still, this tool provides biologists with a slightly
different idea how the real genome looks like, which is expected to help them,
especially considering the fast runtime that it has.

Some of the errors that can be observed when using this approach might be due to
the way the edges are formed. Although it is good that some of the mate-pairs
are bundled, it might lead to incorrectness, as differences in the length, that
might be considerable, have to be unified. It seems that a finer graining of the
edges will still be useful, in order such errors to be avoided. The next
approach that will be presented will try to resolve this issue.
% subsubsection MIR (end)

%\subsubsection{SSPACE} % (fold)
%\label{ssub:SSPACE}
%\textcolor{red}{FIX ME}

%Another approach that again relies very much on heuristics is presented in
%\cite{SSPACE}. The article is noticeable for a few reasons. It is one of the
%first that propose data filtering before the start of the scaffolding.
%Mate-pairs with size outside some predefined by the user constraints are
%filtered, for example.

%Another reason for considering it is that it is the first one, at least to the
%knowledge of the authors of the current article, that provided the possibility
%for multiple input libraries to be used.

%The algorithm otherwise is not very innovative. Starting with the largest
%contig, we try to connect each contig with some of the contigs that are
%connected to it by mate-pairs. If the connection does not violate some
%conditions, it is used. Those conditions are not well described in the paper,
%but it seems as a fast check for compatibility between the ordering so far and
%that implied by the new mate-pair.
%% subsubsection SSPACE (end)

\subsubsection{Opera} % (fold)
\label{ssub:Opera}
As it was already pointed out, the system that is described here manages to
improve the representation of the graph\footnote{Please note that as accuracy is
improved, performance is lost, which makes it hard to tell if this is an
improvement. The author of the current article, however, believe that it is an
improvement.}. Although the article that describes the system Opera
(\cite{Opera}) is presented approximately at the same time as the one that
describes the MIR system - (\cite{MIR}) and the two have some similarities, they
also have some very important differences.

It can be inferred from the description so far that Opera uses again a graph
representation. It again represent contigs as nodes and edges are sets of
mate-pairs that are compatible (suggest the same orientation) and have similar
length (this is the difference with the representation in MIR).

The data is filtered using a different approach from the previous methods that
have been described already. Instead of having a threshold for edge removal
that is given as a parameter, the algorithm tries, using a heuristic approach,
to find the optimal such threshold and trim the edges that have lower support.

The problem definition of the system is very similar to the previous - maximize
the number of edges, for which the contigs they connect are oriented as
suggested by the edges. This problem definition might not be the best one, as it
suggests that removal of edges are equivalent, which does not seem right,
because it neglects the fact that those edges might have different support.

This definition, however, allows the authors to propose an exact method, which
does not use any heuristics. The observation which allows them to accomplish
that is the fact that there is a minimal contig size and a maximum mate-pair size.
This allows them to define a property called \emph{width of the library}, which
is the fraction of the two. It represents how many contigs can at maximum an
edge span over. As this number is relatively small, they suggest to accept it is
an constant.

After that a number of lemmas and theorems are proved that show that there is an
algorithm that can find \emph{p} edges, which should be removed in order the
resulting graph to be consistent, if this is possible, or return that it is
impossible. The runtime of this algorithm is $O(|V|^w |E|^{p + 1})$, where
\emph{w} is the width of the library.

The idea of this algorithm is to use a dynamic programming approach that
exploits classes of equivalence of sub-scaffolds and tries to find an optimal
scaffold, without verifying sub-scaffolds from the same class twice. Although
the theoretical runtime seems high, the runtime of the algorithm is very good
for the practical cases for which the authors have provided information. This
might be thanks to an additional optimization that the authors have suggested.
It is again based on graph partitioning, however this time it is slightly
different. See \cite{Opera} for details.

The results from this technique for smaller organisms are very good. It
outperforms its competitors with a significant amount, at least in some of the
cases. Despite that, its good results might be possible only for small genomes,
as the runtime of the algorithm will become infeasible for big genomes and the
situation will become even worse when the more complex structures on those
bigger genomes are also taken into account.

Having said this, however, the system presented a number of innovations that are
useful even on their own. It also provides a very precise method for small
genome assembly, which is also very good.
% subsubsection Opera (end)

Although the previous systems do have some nice properties, there is at least
one thing that, despite being important, is never considered so far. This is the
fact that contigs can be assembled with errors that might not be so easy to find
using the filters. Those errors might be avoided in the scaffolding phase, but
then the value of the function that is optimized is not the most precise one, as
the removal of all the mate-pairs to an incorrect contig is much more valuable
than removing a few mate-pairs from other contigs. This does not model the real
world very precisely. It should be considered that contigs might be assembled
with errors and the removal of such contigs should not imply very big punishment
for the function that is optimized. Trying to deal with this problem is the
system SCARPA presented in \cite{SCARPA}.

\subsubsection{SCARPA} % (fold)
\label{ssub:SCARPA}
The SCARPA system has similar structure to the those described so far. It first
filters the data. Then tries to orient the contigs. After that it order them and
finally it calculates their final positions in the scaffolding.

The problem definition that is used for the orientation step not only states
that it tries to minimize the number of mate-pairs that are removed (or are not
satisfied), but also considers the removal of entire contigs. The approach for
solving this problem is to first try to remove any cycle that has an odd number
of nodes in it. Then a consistent orientation can be assigned to each
contig. A breadth-first search is performed on the graph to propagate
orientation of nodes in the following manner. First we determine the orientation
of some fixed node. Then for every edge that goes from its end, it should lead
to a start of contig and vise versa.

The removal of odd cycles is NP-hard in general, but there are efficient
algorithms that can solve it for a small number of removed nodes. For this case
its runtime is $O(3^k k n m )$, where \emph{k} is the number of nodes to be
removed. The approach of the authors is to try for different values of k,
starting from 0 and incrementing with 1 each time the algorithm fails to find a
solution.

The graph for which this odd cycle removal problem is solved is constructed as
to each contig, two nodes are created, representing its start and end in some
direction. The two are connected with an edge. Edges are also added for a set of
mate-pairs that link some two contigs and that impose the same orientation. The
appropriate nodes in the graph are connected with the edges that represent
mate-pairs.

The ordering step of the algorithm should put the contigs in linear order, which
can be achieved by removing any cycles left in the graph. An additional
requirement is that the smallest number of edges have to be removed so that this
goal is accomplished. Although this problem is also NP-hard on the general,
there is a heuristic algorithm that solves it for sparse graphs, which
guarantees an asymptotically optimal solution.

The actual positioning is solved as optimization on topological ordering,
minimizing the difference between the expected distances between contigs and the
actual ones, using linear programming technique, followed by a procedure which
detects overlapping between the assigned positions of the contigs and tries to
resolve it.

In order to optimize the performance, the authors have decided to split the
graph in parts that can be separately solved and then combined. To assure that
the graph splitting will be possible, the most common reason for unsplittable
graphs is removed - contigs with a very high number of mate-pairs. Those contigs
are removed, so that the performance of the system do not suffer.

To summarize, the system described in this section, despite looking very similar
to some to some of the already presented systems, has some very big differences.
It tries to solve the problem of too 'costly' removal of incorrectly assembled
contigs from the scaffold. It also provides a nice way for contig orientation,
when there are no odd cycles. The results that it provides are also very nice,
as, even though the accuracy of the assembly is not as high as that of some
others, it consistently provides scaffolds with good quality and longer in size
compared to most of its competitors.

As a drawback again it can be pointed out that it is likely that for bigger data
sets this approach will fail to provide results, as it will be possible, for
example, to have to solve a big optimization problem, which may not feasible for
any solver and that expected contigs coverage is not used in by system.
% subsubsection SCARPA (end)

\subsubsection{GRASS} % (fold)
\label{ssub:GRASS}
The last system to be presented is described in \cite{grass} and is called
GRASS (GeneRic ASsembly Scaffolder). As the name suggests, the idea of this
system is to be more generic than those that have already been presented. The
main difference is that for GRASS the problem is not as much how to use
mate-pairs in order to solve the scaffolding problem. It is how to use a set of
constraints optimally, so that it solves the scaffolding problem optimally. This
means that it is designed to be able to use various data sources simultaneously.
Whether they will be different libraries with various insert sizes or some other
information that can be represented as constraints between pairs of contigs, it
is not that important for that system. Every data source can have different
weight that represent how reliable the information from this data source really
is.

The graph on which the system operates is composed of nodes that represent
contigs and edges that are sets of constraints, that suggest similar
restrictions for the placements of the contigs that they connect.

The optimization criteria is a single quadratic object, which is somewhat
different from what we have seen so far. There are different coefficients that
represent to what extend some constraints are violated. Appropriate penalties
are calculated for such cases.

The optimization is implemented with the use of expectation-maximization (EM)
technique, which iteratively tries to improve the optimized function. The
maximization step supposes that the orientation is known and wants to decide
which links to use. The expectation step tries to find the contig orientation
that maximizes the objective function.

The authors of \cite{grass} show some very interesting results for their system.
Although the results seem impressive, it is peculiar that there is no clear
information about the results of SOPRA. As this system is the most precise one
that is currently available, it would be interesting to see how it will behave
compared to GRASS.

Apart from that little unclear explanation as of why there are no test result
for SOPRA, \cite{grass} presents some very good ideas about how to make a more
generic scaffolding system. Another interesting idea is the use of EM.
% subsubsection GRASS (end)

% subsection Scaffolders that use scaffolding constraints graphs (end)

% section Related word (end)

%*****************************************************************%
\section{Additional steps} % (fold)
\label{sec:Additional steps}
Before and after the scaffolding can occur there are a number of additional
steps that should be presented. Some of them should be clear enough by now,
however, for some of them it might still not be absolutely evident why they are
so important. Here we will try to provide information about that.

\subsection{Data filtering} % (fold)
\label{sub:Data filtering}
The filtering of the data before it is given to the contig assemblers is one of
the critical steps that every good system should execute. There are a number of
tools that can be used for the filtering purpose, including \cite{quake} and
\cite{filtering-solid}. Those systems sometimes exploit information about the
quality of the reads that was written along with the data. This information can
be very useful to detect possibly incorrect reads. There are a number of
additional steps that are taken by the data filtering system, which can improve
significantly the quality of the data with the cost of decrease of the coverage.
The decrease is well compensated for most cases by the ability of the contig
assemblers to assemble correct contigs and the ability of the scaffolders to
take correct decisions based on the correct data and the correct contigs.

As it can be seen from the system descriptions in the previous section, the
problem is so complicated due to the large amount of errors that can be
introduced in different phases of the genome sequencing. If it was not for those
errors, the problem would have much nicer solutions.
% subsection Data filtering (end)

\subsection{The finishing process and gap closing} % (fold)
\label{sub:The finishing process and gap closing}
For a long time biologists have been provided with some computer models of the
genomes of some organisms and those models had to be verified and error and lacks
of information - corrected. This process, called finishing, often is very slow
and costly. As a consequence the scientists have been wondering if it is
possible somehow to improve the quality of the scaffolds that are produced. Such
an effort is \cite{finIs}. It tries to use some of the information that was not
used very much during the scaffold assembly in order to correct some of the
errors that the scaffolder might have made and it also tries to fill any gaps
that might exist in the scaffold.

The information that is most extensively used is the expected number of
occurrences of a contig. A contig that is expected to have occurred multiple
times in the genome can be mapped to just one place of the genome (at least in
general). The same contig may fit well in some of the gaps between the contigs
in the scaffold. This is one of the problems that FinIS tries to resolve.

According to the authors, their approach can accomplish some very nice results,
managing to close as much as 100\% of the gaps on some genomes.
% subsection The finishing process and gap closing (end)

% section Additional steps (end)

%*****************************************************************%
\section{Scaffolder performance evaluation} % (fold)
\label{sec:Scaffolder performance evaluation}
There are a number of criteria that are used to measure the performance of
different scaffolders. One of the them is the size of the scaffolds that are
produced. Naturally, the longer scaffolds, the better. Another criterion that is
used to compare different systems is the number of scaffolds that have been
produced. Naturally, if this number is closer to one, it is better.

Apart from those measurements about the size of the produced scaffolds, there
are others that measure the quality. The most important ones are coverage and
measurements for the errors - most commonly breakpoints. For those measurements
it is essential that the scaffold is mapped to some referential genome, so that
it can be verified that the scaffolding did not perform any error, did not swap
the positions of two contigs, did not put quite different gap between two
contigs than the actual one, etc. One of the most well-known systems for
alignment of scaffolds to referential genome is called MUMmer and is described
in details in \cite{MUMmer}.
% section Scaffolder performance evaluation (end)

%*****************************************************************%
\section{Conclusion} % (fold)
\label{sec:Conclusion}
Although all the presented approaches up until now have some very nice
properties, they all were missing some important pieces of information that
could improve the produced scaffolds and most of all, that could make the
finishing step of the genome assembly easier to a great extend. It is the fact
that there is information about the number of occurrences each contig is about
to have in the scaffolds. As a consequence there are a number of gaps that are
hard to fill after the scaffolding is completed. The resolution of the problem
of the best way to use the expected contig occurrences is far from being
trivial. This is the reason why it has so far been avoided. If you know, for
example, that in some contig it is possible to come from a number of edges and
it is possible to go to another number of edges, it is not clear how to make the
problem equivalent to one of the already solved ones. Each pair of incoming and
outgoing edge can be presented in the optimal scaffolding. Trying all of them is
clearly not a feasible solution.

During the internship, we will try to find a model that would enable the
integration of this information during the scaffolding. An additional
requirement that will be imposed is that the data will not have error (at least
not significant ones). The hope is that it will be possible, when the problem is
solved for the error free case, to be solved for a case where a limited number
of errors are presented, using the same model. Then it would be possible to try
with different error counts, until one for which there is a solution is found
(similarly to the approach in \cite{SCARPA} for odd cycle removal).

Another characteristic that will be required by the system is the ability to
show a number of optimal solutions, possibly all of them. We believe that this
will be very useful for biologists as they will be able to verify the
correctness of our results quite easily.
% section Conclusion (end)

%*****************************************************************%
\begin{thebibliography}{widest entry}
  \bibitem{pevzner-et-all-2001} Pevzner PA, Tang H, Waterman MS. 2001. An
    Eulerian path approach to DNA fragment assembly. Proc Natl Acad Sci 98:
    9748–9753.
  \bibitem{grass} Gritsenko, A. A., Nijkamp, J. F., Reinders, M. J. T.,
    and de Ridder, D. (2012). GRASS: a generic algorithm for scaffolding
    next-generation sequencing assemblies. Bioinformatics, 28(11), 1429–37.
  \bibitem{pevzner} Pevzner Pevzner PA, Tang H: Fragment assembly with
    double-barreled data. Bioinformatics 2001, 17(Suppl 1):S225-233.
  \bibitem{velvet-scaffolding} D.R. Zerbino, G.K. McEwen, E.H. Margulies, E.
    Birney, Pebble and rock band: heuristic resolution of repeats and
    scaffolding in the velvet short-read de novo assembler, PLoS One 4 (2009)
    e8407.
  \bibitem{SOAPdenovo} R. Li, H. Zhu, J. Ruan, W. Qian, X. Fang, Z. Shi, Y. Li,
    S. Li, G. Shan, K. Kristiansen, H. Yang, J. Wang, De novo assembly of
    human genomes with massively parallel short read sequencing, Genome Res. 20
    (2009) 265 - 272.
  \bibitem{SOPRA} Dayarian, A., Michael, T.P. and Sengupta, A.M. (2011) SOPRA:
    scaffolding algorithm for paired reads via statistical optimization, BMC
    Bioinformatics, 11, doi:10.1186/1471-2105-11-345.
  \bibitem{greedy-path-merging} Huson DH, Reinert K, Myers EW: The greedy
    path-merging algorithm for Contig Scaffolding. Journal of the Acm 2002,
    49(5):603-615.
  \bibitem{SSPACE} Boetzer,M. et al. (2011) Scaffolding pre-assembled contigs
    using SSPACE. Bioinformatics, 27, 578–579.
  \bibitem{MIR} Salmela, L., Mäkinen, V., Välimäki, N., Ylinen, J., Ukkonen,
    E. (2011). Fast scaffolding with small independent mixed integer programs.
    Bioinformatics, 27(23), 3259-3265.
  \bibitem{SCARPA} Donmez, N., Brudno, M. (2013). SCARPA: scaffolding reads
    with practical algorithms. Bioinformatics, 29(4), 428-434.
  \bibitem{Opera} Gao,S. et al. (2011) Opera: reconstructing optimal genomic
    scaffolds with high-throughput paired-end sequences. In: Bafna,V. and
    Sahinalp,S. (eds) Research in Computational Molecular Biology. Vol. 6577 of
    Lecture Notes in Computer Science. Springer, Berlin/Heidelberg, pp. 437–451.
  \bibitem{westbrook-tarjan} Westbrook,J. and Tarjan,R.E. (1992) Maintaining
    bridge-connected and biconnected components on-line. Algorithmica, 7,
    433–464.
  \bibitem{quake} Kelley, D. R., Schatz, M. C., Salzberg, S. L. (2010). Quake:
    quality-aware detection and correction of sequencing errors. Genome Biol,
    11(11), R116.
  \bibitem{filtering-solid} Sasson, A., Michael, T. P. (2010). Filtering error
    from SOLiD output. Bioinformatics, 26(6), 849-850.
  \bibitem{finIs} Gao, S., Bertrand, D., Nagarajan, N. (2012). FinIS: improved
    in silico finishing using an exact quadratic programming formulation. In
    Algorithms in Bioinformatics (pp. 314-325). Springer Berlin Heidelberg.
  \bibitem{MUMmer} Delcher, A.L., Phillippy, A., Carlton, J. and Salzberg, S.L.
    (2002) Fast algorithms for large-scale genome alignment and comparison,
    Nucleic Acids Research, 30, 2478– 2483, doi:10.1093/nar/30.11.2478
\end{thebibliography}

\end{document}
